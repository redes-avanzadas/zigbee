
%% bare_jrnl_compsoc.tex
%% V1.4a
%% 2014/09/17
%% by Michael Shell
%% See:
%% http://www.michaelshell.org/
%% for current contact information.
%%
%% This is a skeleton file demonstrating the use of IEEEtran.cls
%% (requires IEEEtran.cls version 1.8a or later) with an IEEE
%% Computer Society journal paper.
%%
%% Support sites:
%% http://www.michaelshell.org/tex/ieeetran/
%% http://www.ctan.org/tex-archive/macros/latex/contrib/IEEEtran/
%% and
%% http://www.ieee.org/

%%*************************************************************************
%% Legal Notice:
%% This code is offered as-is without any warranty either expressed or
%% implied; without even the implied warranty of MERCHANTABILITY or
%% FITNESS FOR A PARTICULAR PURPOSE! 
%% User assumes all risk.
%% In no event shall IEEE or any contributor to this code be liable for
%% any damages or losses, including, but not limited to, incidental,
%% consequential, or any other damages, resulting from the use or misuse
%% of any information contained here.
%%
%% All comments are the opinions of their respective authors and are not
%% necessarily endorsed by the IEEE.
%%
%% This work is distributed under the LaTeX Project Public License (LPPL)
%% ( http://www.latex-project.org/ ) version 1.3, and may be freely used,
%% distributed and modified. A copy of the LPPL, version 1.3, is included
%% in the base LaTeX documentation of all distributions of LaTeX released
%% 2003/12/01 or later.
%% Retain all contribution notices and credits.
%% ** Modified files should be clearly indicated as such, including  **
%% ** renaming them and changing author support contact information. **
%%
%% File list of work: IEEEtran.cls, IEEEtran_HOWTO.pdf, bare_adv.tex,
%%                    bare_conf.tex, bare_jrnl.tex, bare_conf_compsoc.tex,
%%                    bare_jrnl_compsoc.tex, bare_jrnl_transmag.tex
%%*************************************************************************


% *** Authors should verify (and, if needed, correct) their LaTeX system  ***
% *** with the testflow diagnostic prior to trusting their LaTeX platform ***
% *** with production work. IEEE's font choices and paper sizes can       ***
% *** trigger bugs that do not appear when using other class files.       ***                          ***
% The testflow support page is at:
% http://www.michaelshell.org/tex/testflow/


\documentclass[10pt,journal,compsoc]{IEEEtran}
%
% If IEEEtran.cls has not been installed into the LaTeX system files,
% manually specify the path to it like:
% \documentclass[10pt,journal,compsoc]{../sty/IEEEtran}





% Some very useful LaTeX packages include:
% (uncomment the ones you want to load)


% *** MISC UTILITY PACKAGES ***
%
%\usepackage{ifpdf}
% Heiko Oberdiek's ifpdf.sty is very useful if you need conditional
% compilation based on whether the output is pdf or dvi.
% usage:
% \ifpdf
%   % pdf code
% \else
%   % dvi code
% \fi
% The latest version of ifpdf.sty can be obtained from:
% http://www.ctan.org/tex-archive/macros/latex/contrib/oberdiek/
% Also, note that IEEEtran.cls V1.7 and later provides a builtin
% \ifCLASSINFOpdf conditional that works the same way.
% When switching from latex to pdflatex and vice-versa, the compiler may
% have to be run twice to clear warning/error messages.






% *** CITATION PACKAGES ***
%
\ifCLASSOPTIONcompsoc
  % IEEE Computer Society needs nocompress option
  % requires cite.sty v4.0 or later (November 2003)
  \usepackage[nocompress]{cite}
\else
  % normal IEEE
  \usepackage{cite}
\fi
% cite.sty was written by Donald Arseneau
% V1.6 and later of IEEEtran pre-defines the format of the cite.sty package
% \cite{} output to follow that of IEEE. Loading the cite package will
% result in citation numbers being automatically sorted and properly
% "compressed/ranged". e.g., [1], [9], [2], [7], [5], [6] without using
% cite.sty will become [1], [2], [5]--[7], [9] using cite.sty. cite.sty's
% \cite will automatically add leading space, if needed. Use cite.sty's
% noadjust option (cite.sty V3.8 and later) if you want to turn this off
% such as if a citation ever needs to be enclosed in parenthesis.
% cite.sty is already installed on most LaTeX systems. Be sure and use
% version 5.0 (2009-03-20) and later if using hyperref.sty.
% The latest version can be obtained at:
% http://www.ctan.org/tex-archive/macros/latex/contrib/cite/
% The documentation is contained in the cite.sty file itself.
%
% Note that some packages require special options to format as the Computer
% Society requires. In particular, Computer Society  papers do not use
% compressed citation ranges as is done in typical IEEE papers
% (e.g., [1]-[4]). Instead, they list every citation separately in order
% (e.g., [1], [2], [3], [4]). To get the latter we need to load the cite
% package with the nocompress option which is supported by cite.sty v4.0
% and later. Note also the use of a CLASSOPTION conditional provided by
% IEEEtran.cls V1.7 and later.





% *** GRAPHICS RELATED PACKAGES ***
%
\ifCLASSINFOpdf
  % \usepackage[pdftex]{graphicx}
  % declare the path(s) where your graphic files are
  % \graphicspath{{../pdf/}{../jpeg/}}
  % and their extensions so you won't have to specify these with
  % every instance of \includegraphics
  % \DeclareGraphicsExtensions{.pdf,.jpeg,.png}
\else
  % or other class option (dvipsone, dvipdf, if not using dvips). graphicx
  % will default to the driver specified in the system graphics.cfg if no
  % driver is specified.
  % \usepackage[dvips]{graphicx}
  % declare the path(s) where your graphic files are
  % \graphicspath{{../eps/}}
  % and their extensions so you won't have to specify these with
  % every instance of \includegraphics
  % \DeclareGraphicsExtensions{.eps}
\fi
% graphicx was written by David Carlisle and Sebastian Rahtz. It is
% required if you want graphics, photos, etc. graphicx.sty is already
% installed on most LaTeX systems. The latest version and documentation
% can be obtained at: 
% http://www.ctan.org/tex-archive/macros/latex/required/graphics/
% Another good source of documentation is "Using Imported Graphics in
% LaTeX2e" by Keith Reckdahl which can be found at:
% http://www.ctan.org/tex-archive/info/epslatex/
%
% latex, and pdflatex in dvi mode, support graphics in encapsulated
% postscript (.eps) format. pdflatex in pdf mode supports graphics
% in .pdf, .jpeg, .png and .mps (metapost) formats. Users should ensure
% that all non-photo figures use a vector format (.eps, .pdf, .mps) and
% not a bitmapped formats (.jpeg, .png). IEEE frowns on bitmapped formats
% which can result in "jaggedy"/blurry rendering of lines and letters as
% well as large increases in file sizes.
%
% You can find documentation about the pdfTeX application at:
% http://www.tug.org/applications/pdftex






% *** MATH PACKAGES ***
%
%\usepackage[cmex10]{amsmath}
% A popular package from the American Mathematical Society that provides
% many useful and powerful commands for dealing with mathematics. If using
% it, be sure to load this package with the cmex10 option to ensure that
% only type 1 fonts will utilized at all point sizes. Without this option,
% it is possible that some math symbols, particularly those within
% footnotes, will be rendered in bitmap form which will result in a
% document that can not be IEEE Xplore compliant!
%
% Also, note that the amsmath package sets \interdisplaylinepenalty to 10000
% thus preventing page breaks from occurring within multiline equations. Use:
%\interdisplaylinepenalty=2500
% after loading amsmath to restore such page breaks as IEEEtran.cls normally
% does. amsmath.sty is already installed on most LaTeX systems. The latest
% version and documentation can be obtained at:
% http://www.ctan.org/tex-archive/macros/latex/required/amslatex/math/





% *** SPECIALIZED LIST PACKAGES ***
%
%\usepackage{algorithmic}
% algorithmic.sty was written by Peter Williams and Rogerio Brito.
% This package provides an algorithmic environment fo describing algorithms.
% You can use the algorithmic environment in-text or within a figure
% environment to provide for a floating algorithm. Do NOT use the algorithm
% floating environment provided by algorithm.sty (by the same authors) or
% algorithm2e.sty (by Christophe Fiorio) as IEEE does not use dedicated
% algorithm float types and packages that provide these will not provide
% correct IEEE style captions. The latest version and documentation of
% algorithmic.sty can be obtained at:
% http://www.ctan.org/tex-archive/macros/latex/contrib/algorithms/
% There is also a support site at:
% http://algorithms.berlios.de/index.html
% Also of interest may be the (relatively newer and more customizable)
% algorithmicx.sty package by Szasz Janos:
% http://www.ctan.org/tex-archive/macros/latex/contrib/algorithmicx/




% *** ALIGNMENT PACKAGES ***
%
%\usepackage{array}
% Frank Mittelbach's and David Carlisle's array.sty patches and improves
% the standard LaTeX2e array and tabular environments to provide better
% appearance and additional user controls. As the default LaTeX2e table
% generation code is lacking to the point of almost being broken with
% respect to the quality of the end results, all users are strongly
% advised to use an enhanced (at the very least that provided by array.sty)
% set of table tools. array.sty is already installed on most systems. The
% latest version and documentation can be obtained at:
% http://www.ctan.org/tex-archive/macros/latex/required/tools/


% IEEEtran contains the IEEEeqnarray family of commands that can be used to
% generate multiline equations as well as matrices, tables, etc., of high
% quality.




% *** SUBFIGURE PACKAGES ***
%\ifCLASSOPTIONcompsoc
%  \usepackage[caption=false,font=footnotesize,labelfont=sf,textfont=sf]{subfig}
%\else
%  \usepackage[caption=false,font=footnotesize]{subfig}
%\fi
% subfig.sty, written by Steven Douglas Cochran, is the modern replacement
% for subfigure.sty, the latter of which is no longer maintained and is
% incompatible with some LaTeX packages including fixltx2e. However,
% subfig.sty requires and automatically loads Axel Sommerfeldt's caption.sty
% which will override IEEEtran.cls' handling of captions and this will result
% in non-IEEE style figure/table captions. To prevent this problem, be sure
% and invoke subfig.sty's "caption=false" package option (available since
% subfig.sty version 1.3, 2005/06/28) as this is will preserve IEEEtran.cls
% handling of captions.
% Note that the Computer Society format requires a sans serif font rather
% than the serif font used in traditional IEEE formatting and thus the need
% to invoke different subfig.sty package options depending on whether
% compsoc mode has been enabled.
%
% The latest version and documentation of subfig.sty can be obtained at:
% http://www.ctan.org/tex-archive/macros/latex/contrib/subfig/




% *** FLOAT PACKAGES ***
%
%\usepackage{fixltx2e}
% fixltx2e, the successor to the earlier fix2col.sty, was written by
% Frank Mittelbach and David Carlisle. This package corrects a few problems
% in the LaTeX2e kernel, the most notable of which is that in current
% LaTeX2e releases, the ordering of single and double column floats is not
% guaranteed to be preserved. Thus, an unpatched LaTeX2e can allow a
% single column figure to be placed prior to an earlier double column
% figure. The latest version and documentation can be found at:
% http://www.ctan.org/tex-archive/macros/latex/base/


%\usepackage{stfloats}
% stfloats.sty was written by Sigitas Tolusis. This package gives LaTeX2e
% the ability to do double column floats at the bottom of the page as well
% as the top. (e.g., "\begin{figure*}[!b]" is not normally possible in
% LaTeX2e). It also provides a command:
%\fnbelowfloat
% to enable the placement of footnotes below bottom floats (the standard
% LaTeX2e kernel puts them above bottom floats). This is an invasive package
% which rewrites many portions of the LaTeX2e float routines. It may not work
% with other packages that modify the LaTeX2e float routines. The latest
% version and documentation can be obtained at:
% http://www.ctan.org/tex-archive/macros/latex/contrib/sttools/
% Do not use the stfloats baselinefloat ability as IEEE does not allow
% \baselineskip to stretch. Authors submitting work to the IEEE should note
% that IEEE rarely uses double column equations and that authors should try
% to avoid such use. Do not be tempted to use the cuted.sty or midfloat.sty
% packages (also by Sigitas Tolusis) as IEEE does not format its papers in
% such ways.
% Do not attempt to use stfloats with fixltx2e as they are incompatible.
% Instead, use Morten Hogholm'a dblfloatfix which combines the features
% of both fixltx2e and stfloats:
%
% \usepackage{dblfloatfix}
% The latest version can be found at:
% http://www.ctan.org/tex-archive/macros/latex/contrib/dblfloatfix/




%\ifCLASSOPTIONcaptionsoff
%  \usepackage[nomarkers]{endfloat}
% \let\MYoriglatexcaption\caption
% \renewcommand{\caption}[2][\relax]{\MYoriglatexcaption[#2]{#2}}
%\fi
% endfloat.sty was written by James Darrell McCauley, Jeff Goldberg and 
% Axel Sommerfeldt. This package may be useful when used in conjunction with 
% IEEEtran.cls'  captionsoff option. Some IEEE journals/societies require that
% submissions have lists of figures/tables at the end of the paper and that
% figures/tables without any captions are placed on a page by themselves at
% the end of the document. If needed, the draftcls IEEEtran class option or
% \CLASSINPUTbaselinestretch interface can be used to increase the line
% spacing as well. Be sure and use the nomarkers option of endfloat to
% prevent endfloat from "marking" where the figures would have been placed
% in the text. The two hack lines of code above are a slight modification of
% that suggested by in the endfloat docs (section 8.4.1) to ensure that
% the full captions always appear in the list of figures/tables - even if
% the user used the short optional argument of \caption[]{}.
% IEEE papers do not typically make use of \caption[]'s optional argument,
% so this should not be an issue. A similar trick can be used to disable
% captions of packages such as subfig.sty that lack options to turn off
% the subcaptions:
% For subfig.sty:
% \let\MYorigsubfloat\subfloat
% \renewcommand{\subfloat}[2][\relax]{\MYorigsubfloat[]{#2}}
% However, the above trick will not work if both optional arguments of
% the \subfloat command are used. Furthermore, there needs to be a
% description of each subfigure *somewhere* and endfloat does not add
% subfigure captions to its list of figures. Thus, the best approach is to
% avoid the use of subfigure captions (many IEEE journals avoid them anyway)
% and instead reference/explain all the subfigures within the main caption.
% The latest version of endfloat.sty and its documentation can obtained at:
% http://www.ctan.org/tex-archive/macros/latex/contrib/endfloat/
%
% The IEEEtran \ifCLASSOPTIONcaptionsoff conditional can also be used
% later in the document, say, to conditionally put the References on a 
% page by themselves.




% *** PDF, URL AND HYPERLINK PACKAGES ***
%
\usepackage{url}
% url.sty was written by Donald Arseneau. It provides better support for
% handling and breaking URLs. url.sty is already installed on most LaTeX
% systems. The latest version and documentation can be obtained at:
% http://www.ctan.org/tex-archive/macros/latex/contrib/url/
% Basically, \url{my_url_here}.

\usepackage[utf8]{inputenc}
\usepackage[T1]{fontenc}
\usepackage[spanish]{babel}
\usepackage{graphicx}
\usepackage{booktabs}
\graphicspath{ {images/} }



% *** Do not adjust lengths that control margins, column widths, etc. ***
% *** Do not use packages that alter fonts (such as pslatex).         ***
% There should be no need to do such things with IEEEtran.cls V1.6 and later.
% (Unless specifically asked to do so by the journal or conference you plan
% to submit to, of course. )


% correct bad hyphenation here
\hyphenation{op-tical net-works semi-conduc-tor}


\begin{document}
%
% paper title
% Titles are generally capitalized except for words such as a, an, and, as,
% at, but, by, for, in, nor, of, on, or, the, to and up, which are usually
% not capitalized unless they are the first or last word of the title.
% Linebreaks \\ can be used within to get better formatting as desired.
% Do not put math or special symbols in the title.
\title{ZigBee}
%
%
% author names and IEEE memberships
% note positions of commas and nonbreaking spaces ( ~ ) LaTeX will not break
% a structure at a ~ so this keeps an author's name from being broken across
% two lines.
% use \thanks{} to gain access to the first footnote area
% a separate \thanks must be used for each paragraph as LaTeX2e's \thanks
% was not built to handle multiple paragraphs
%
%
%\IEEEcompsocitemizethanks is a special \thanks that produces the bulleted
% lists the Computer Society journals use for "first footnote" author
% affiliations. Use \IEEEcompsocthanksitem which works much like \item
% for each affiliation group. When not in compsoc mode,
% \IEEEcompsocitemizethanks becomes like \thanks and
% \IEEEcompsocthanksitem becomes a line break with idention. This
% facilitates dual compilation, although admittedly the differences in the
% desired content of \author between the different types of papers makes a
% one-size-fits-all approach a daunting prospect. For instance, compsoc 
% journal papers have the author affiliations above the "Manuscript
% received ..."  text while in non-compsoc journals this is reversed. Sigh.



% note the % following the last \IEEEmembership and also \thanks - 
% these prevent an unwanted space from occurring between the last author name
% and the end of the author line. i.e., if you had this:
% 
% \author{....lastname \thanks{...} \thanks{...} }
%                     ^------------^------------^----Do not want these spaces!
%
% a space would be appended to the last name and could cause every name on that
% line to be shifted left slightly. This is one of those "LaTeX things". For
% instance, "\textbf{A} \textbf{B}" will typeset as "A B" not "AB". To get
% "AB" then you have to do: "\textbf{A}\textbf{B}"
% \thanks is no different in this regard, so shield the last } of each \thanks
% that ends a line with a % and do not let a space in before the next \thanks.
% Spaces after \IEEEmembership other than the last one are OK (and needed) as
% you are supposed to have spaces between the names. For what it is worth,
% this is a minor point as most people would not even notice if the said evil
% space somehow managed to creep in.

\author{\IEEEauthorblockN{Priscilla~Piedra y Martín~Flores}\\
        \IEEEauthorblockA{
        Escuela de Ingeniería en Computación\\
        Tecnológico de Costa Rica. Cartago, Costa Rica
        }\\
        \small{\texttt{\{ppiedra90, mfloresg\}}\texttt{@gmail.com}}% <-this % stops a space
\thanks{Este documento fue realizado durante el curso Redes de Computadoras Avanzadas, impartido por el profesor Luis Carlos Loaiza Canet. Programa de Maestría en Computación, Instituto Tecnológico de Costa Rica. Segundo Semestre, 2017.}
}

% The paper headers
%\markboth{Journal of \LaTeX\ Class Files,~Vol.~13, No.~9, September~2014}%
%{Shell \MakeLowercase{\textit{et al.}}: Bare Demo of IEEEtran.cls for Computer Society Journals}

\markboth{Redes de Computadoras Avanzadas, Setiembre 2017}%
{Shell \MakeLowercase{\textit{et al.}}: Bare Demo of IEEEtran.cls for Computer Society Journals}


% The only time the second header will appear is for the odd numbered pages
% after the title page when using the twoside option.
% 
% *** Note that you probably will NOT want to include the author's ***
% *** name in the headers of peer review papers.                   ***
% You can use \ifCLASSOPTIONpeerreview for conditional compilation here if
% you desire.



% The publisher's ID mark at the bottom of the page is less important with
% Computer Society journal papers as those publications place the marks
% outside of the main text columns and, therefore, unlike regular IEEE
% journals, the available text space is not reduced by their presence.
% If you want to put a publisher's ID mark on the page you can do it like
% this:
%\IEEEpubid{0000--0000/00\$00.00~\copyright~2014 IEEE}
% or like this to get the Computer Society new two part style.
%\IEEEpubid{\makebox[\columnwidth]{\hfill 0000--0000/00/\$00.00~\copyright~2014 IEEE}%
%\hspace{\columnsep}\makebox[\columnwidth]{Published by the IEEE Computer Society\hfill}}
% Remember, if you use this you must call \IEEEpubidadjcol in the second
% column for its text to clear the IEEEpubid mark (Computer Society jorunal
% papers don't need this extra clearance.)



% use for special paper notices
%\IEEEspecialpapernotice{(Invited Paper)}



% for Computer Society papers, we must declare the abstract and index terms
% PRIOR to the title within the \IEEEtitleabstractindextext IEEEtran
% command as these need to go into the title area created by \maketitle.
% As a general rule, do not put math, special symbols or citations
% in the abstract or keywords.
\IEEEtitleabstractindextext{%
\begin{abstract}
ZigBee es un estándar basado en IEEE 802.15.4 para redes de área personal. Este estándar permite la creacio4n de redes de bajo costo y consumo de energía. Estas redes son creadas a partir de sensores y dispositivos de bateria limitada y pueden controlar inalámbricamente muchos productos eléctricos tales como controles remotos, sensores médicos, industriales y de seguridad. En este reporte se brinda una introducción al estándar y a sus conceptos principales. 



\end{abstract}

% Note that keywords are not normally used for peerreview papers.
%\begin{IEEEkeywords}
%Computer Society, IEEEtran, journal, \LaTeX, paper, template.
%\end{IEEEkeywords}
}


% make the title area
\maketitle


% To allow for easy dual compilation without having to reenter the
% abstract/keywords data, the \IEEEtitleabstractindextext text will
% not be used in maketitle, but will appear (i.e., to be "transported")
% here as \IEEEdisplaynontitleabstractindextext when the compsoc 
% or transmag modes are not selected <OR> if conference mode is selected 
% - because all conference papers position the abstract like regular
% papers do.
\IEEEdisplaynontitleabstractindextext
% \IEEEdisplaynontitleabstractindextext has no effect when using
% compsoc or transmag under a non-conference mode.



% For peer review papers, you can put extra information on the cover
% page as needed:
% \ifCLASSOPTIONpeerreview
% \begin{center} \bfseries EDICS Category: 3-BBND \end{center}
% \fi
%
% For peerreview papers, this IEEEtran command inserts a page break and
% creates the second title. It will be ignored for other modes.
\IEEEpeerreviewmaketitle



\IEEEraisesectionheading{\section{Introducción}\label{sec:introduction}}
% Computer Society journal (but not conference!) papers do something unusual
% with the very first section heading (almost always called "Introduction").
% They place it ABOVE the main text! IEEEtran.cls does not automatically do
% this for you, but you can achieve this effect with the provided
% \IEEEraisesectionheading{} command. Note the need to keep any \label that
% is to refer to the section immediately after \section in the above as
% \IEEEraisesectionheading puts \section within a raised box.




% The very first letter is a 2 line initial drop letter followed
% by the rest of the first word in caps (small caps for compsoc).
% 
% form to use if the first word consists of a single letter:
% \IEEEPARstart{A}{demo} file is ....
% 
% form to use if you need the single drop letter followed by
% normal text (unknown if ever used by IEEE):
% \IEEEPARstart{A}{}demo file is ....
% 
% Some journals put the first two words in caps:
% \IEEEPARstart{T}{his demo} file is ....
% 
% Here we have the typical use of a "T" for an initial drop letter
% and "HIS" in caps to complete the first word.
\IEEEPARstart{L}{as}


\section{ZigBee}

\textbf{PRISCILLA}
%Yo ya tenia una parte introductoria desarrollada y no la quise perder asi que se la puse a su parte de introducción. Considero que no está completa era solo para iniciar.

ZigBee es un estándar desarrollado por la \emph{ZigBee Alliance} para redes de área personal (PAN por sus siglas en inglés). Conformada por más de 270 compañías, la \emph{ZigBee Alliance} es un consorcio que promueve el estándar ZigBee para sensores inalámbricos y dispositivos de red de baja potencia y velocidad. El protocolo ZigBee está construido por encima del protocolo IEEE 802.15.4 el cual define la dirección MAC y las capas físicas para redes de área personal inalámbricas de bajo consumo (LR-WPAN). El estándar ZigBee ofrece un conjunto de perfiles que definen las capas de red, seguridad y aplicación. Los desarrolladores son responsables de crear sus propios perfiles de aplicación o de la integración con los perfiles públicos que fueron desarrollados por la \emph{ZigBee Alliance}. La especificación ZigBee es un estándar abierto que permite a los fabricantes desarrollar sus propias aplicaciones específicas que requieran de bajo costo y consumo de energía.

La especificación ZigBee ha sufrido varias modificaciones desde la publicación de la primera especificación en el 2004:
\begin{itemize}
    \item En el 2004, la \emph{ZigBee Alliance} publica la primera especificación, la cual daba soporte a un perfil de iluminación de control doméstico. Sin embargo, la \emph{Zigbee Alliance} ya no da soporte a la especificación del 2004.
    \item En Febrero del 2006, la \emph{ZigBee Alliance} publicón el \emph{ZigBee Stack 2016}, el cual contenía modificaciones del ZigBee 2004.
    \item En Octubre del 2007, la \emph{ZigBee Alliance} publica dos conjuntos de funcionalidades llamados ZigBee y ZigBee PRO. El conjunto de funcionalidad ZigBee es interoperable con ZigBee PRO. Si una red está basada en el ZigBee PRO, los dispostivos de ZigBee pueden unirse a la red como dispositivos finales (\emph{end devices}). De la misma forma si una red está basada en ZigBee, los dispositivos ZigBee PRO puede unirse a la red como dispositivos finales.
\end{itemize}

\subsection{Características}

\paragraph*{Confianza} A pesar de que las redes wireless son poco confiables donde factores como el ambiente, altura, ubicación representan problemas de comunicación a causa de una recepción pobre. ZigBee provee mediante protocolos (IEEE 802.15.4, CSMA-CA, etc) confianza en la red. Para aumentar la misma ZigBee implementa un CSMA-CA (\emph{Carries Sense Multiple Access Collition Avoidance}) que establece el proceso de comunicación: antes de transmitir ZigBee escucha el canal, cuando el canal esta claro ZigBee comienza a transmitir.



\section{Red ZigBee}
\subsection{Estándar  802.15.4}

La IEEE 802.15.4 es una tecnología wireless  estándar para redes de bajo alcance moderna y robusta creada por la IEEE que utiliza O-QPSK (Offset-Quadature Phase-Shift Keying) y DSSS (Direct Sequence Spread Spectrum) las cuales son tecnologías que proveen un buen rendimiento en ambientes de baja señal. Propone ofrecer capas fundamentales de red wirless WPAN que se basa en el bajo costo, baja velocidad ubicua entre dispositivos. Se enfoca en el bajo costo de comunicación entre dispositivos cercanos permitiendo una cobertura de 10 metros de rango con un rango de transferencia de 250 kbits/s. 

El estándar define dos tipos de nodos para las redes: FFD (full-function device) el cual puede servir como coordinador de una área de network personal, implementa un modelo general de comunicación que permite establecer una relación entre otros dispositivos y RFD (reduced-function devices) el cual se implementa en dispositivos simples con recursos y requerimientos de comunicación modestos, por esta razón no pueden tener el rol de coordinadores. 

\begin{figure}[h]
    \centering
    \includegraphics[width=8cm]{diseno_redes}
    \caption{Diseño de redes. Fuente: \emph{Wikipedia}}
    \label{fig:sdn-arquitectura}
\end{figure}

El estándar establece dos formas de diseñar redes: peer-to-peer y de estrella. Toda red ocupa por lo menos un coordinador FFD. Las redes peer-to-peer pueden formar patrones arbitrarios de conexiones y la extinción es solo limitada por la distancia entre los pares de nodos. Por otro lado una red estrella es mas estructurada donde el coordinador de la red necesita ser el nodo central.

El atractivo de ZigBee es la confianza que brinda a pesar de que construye una red wirless y parte del porque es la implementación del estándar IEEE 802.15.4. Al ser un estándar hay certificaciones de dicho estándar y de ZigBee lo cual le brinda a la comunidad un ambiente robusto para el desarrollo de este tipo de redes.


También ZigBee implementa una solución de dos SAPs por capa una para la data y otra para la administración, esto según las especificaciones del protocolo 802.15.4, lo mismo ocurre en las capas inferiores MAC y PHY donde se define un radio de 2.4 GHz como base de las comunicaciones que permite, por ejemplo un nodo que controla o monitorea un grupo de interruptores solo va a necesitar un radio 802.15.4. (SAPs).

\subsection{Tipos de Red}

\textbf{Priscilla lo hace}

\subsection{Tipos de dispositivos}
Una red ZigBee consiste de nodos ZigBee (dispositivos). La arquitectura de un nodo se muestra en la figura \ref{fig:device-architecture}. Un nodo consiste de un microcontrolador, un tranceptor (\emph{transciver}) y una antena. Un nodo ZigBee usa conjuntos de perfiles los cuales son desarrollados en software. Un nodo puede ser usado para una amplia variedad de aplicaciones, por ejempo, control de la iluminación, detectores de humo y monitoreo doméstico de seguridad. Por lo tanto, un nodo puede soportar múltiples subunidades y cada subunidad tiene un objecto de aplicación que describe la función de la subunidad. Un nodo puede operar ya sea como un \emph{full-function device} (FFD) o como un \emph{reduced-function device} (RFD). Un FFD puede realizar todas las tareas que están definidades por el estándar ZigBee y, opera en todo el conjunto de la capa MAC de IEEE 802.15.4. Un RFD realiza solo un número limitado de tareas.

\begin{figure}[h]
    \centering
    \includegraphics[width=8cm]{device-architecture}
    \caption{Arquitectura de un nodo ZigBee}
    \label{fig:device-architecture}
\end{figure}

\begin{itemize}
    \item \textbf{Coordinador:} un coordinador es un FFD responsable de la gestión global de la red. Cada red tiene exactamente un coordinador. El coordinador realiza las siguientes funciones:
    \begin{itemize}
        \item Selecciona el canal que va a ser usado por la red
        \item Inicia la red
        \item Asigna cómo las direcciones serán reservadas por nodos o \emph{routers}
        \item Permite a otros dispositivos unirse o dejar la red
        \item Mantiene una lista de vecinos y \emph{routers}
        \item Transfiere paquetes de aplicación
    \end{itemize}
    \item \textbf{Dispositivo final (\emph{End device:})} un dispositivo final puede ser un RFD. Un RFD opera dentro de un conjunto limitado de la capa MAC de IEEE 802.15.4, permitiéndole consumir menor energía. El dispositivo final (\emph{child}) se puede conectar a un \emph{router} o a un coordinador (\emph{parent}). También opera a una baja potencia de ciclo de trabajo, lo que significa que consume energía solo mientras transmite información. Por lo tanto, la arquitectura ZigBee está diseñada de tal forma que el tiempo de transmisión de dispositivo final sea corto. Un dispositivo final realiza las siguientes funciones:
    \begin{itemize}
        \item Se une o deja una red
        \item Transfiere paquetes de aplicación
    \end{itemize}
    \item \textbf{\emph{Router}:} un \emph{router} es un FFD. Un \emph{router} es usado en un topologías de árbol y malla (\emph{mesh}) para expandir la cobertura de la red. La función de un \emph{router} es encontrar la mejor ruta al destino para transferir un mensaje. Un \emph{router} realiza todas las funciones similares de un coordinador excepto el establecimiento de una red.
    \item \textbf{\emph{ZigBee trust center (ZTC)}:} el ZTC es un dispositivo que provee gestión de seguridad, distribución segura de llaves y autenticación de dispositivos.
    \item \textbf{\emph{ZigBee gateway}:} El \emph{ZigBee gateway} es usado para conectar la red ZigBee con otra red tal y como una LAN, por medio de la conversión de protocolos.
\end{itemize}


\subsection{Topologias}
ZigBee usa la especificación 802.15.4 para su capa física y capa MAC. IEEE 802.15.4 ofrece topologías de estrella, árbol, \emph{cluster tree} y malla; sin embargo, ZigBee soporta solo topologías de estrella, árbol y malla.

Se utiliza una jerarquía de asociación; un dispositivo que se une a la red puede ser \emph{router} o un dispositivo final, y los \emph{routers} pueden aceptar más dispositivos.

\begin{itemize}
    \item \textbf{Topología de estrella:} La topología de estrella consiste de un coordinador y varios dispositivos (nodos), como se muestra en la figura \ref{fig:star-topology}. En esta topología, el dispositivo final se comunica solamente con el coordinador. Cualquier intercambio de paquetes entre dispositivos finales debe ir a través del coordinador. La desventaja de esta topología es que la operación de la red depende del coordinador de la red y, debido a que todos los paquetes entre los dispositivos deben de pasar a través del coordinador, el coordinador se convierte en un cuello de botella. También, no hay ruta alterna desde el origen hacia el destino. La ventaja de la topología de estrella es que es simple y que los paquetes viajan realizan a lo sumo dos saltos para llegar a su destino.
    \begin{figure}[h]
        \centering
        \includegraphics[width=7.5cm]{star-topology}
        \caption{Topología de estrella}
        \label{fig:star-topology}
    \end{figure}
    \item \textbf{Topología de árbol:} En esta topología, la red consiste de un nodo central (nodo raíz), el cual es un coordinador, varios \emph{routers} y dispositivos finales tal y como se muestra en la figura \ref{fig:tree-topology}. La función del \emph{router} es la de extender la cobertura de la red. Los nodos finales que están conectados al coordinador o a los \emph{routers} se les llama \emph{children}. Solamente los \emph{routers} y el coordinador pueden tener  hijos \emph{children}. Cada dispositivo final está solamente en la capacidad de comunicarse con su padre (\emph{router} o coordinador). El coordinador y los \emph{routers} puede tener hijos y, por tanto, son los únicos dispositivos que pueden ser padres. Un dispositivo final no puede tener hijos y por lo tanto, no puede ser un padre. Un caso especial de topología de árbol es la llamada topología \emph{cluster tree}.
    \begin{figure}[h]
        \centering
        \includegraphics[width=7.5cm]{tree-topology}
        \caption{Topología de árbol}
        \label{fig:tree-topology}
    \end{figure}    
    Las desventajas de una topología de árbol son:
    \begin{itemize}
        \item[a.] Si uno de los padres está deshabilitado, los hijos del padre deshabilitado no van a poder comunicarse con otros dispositivos en la red. 
        \item[b.] Incluso si dos nodos están geográficamente cerca el uno del otro, estos no se van a poder comunicar directamente.
    \end{itemize}
    \item \textbf{\emph{Cluster tree topology}:} Una topología \emph{cluster tree} es un caso especial de la topología de árbol en donde a un padre con sus hijos se le llama un \emph{cluster}, tal y como se muestra en la figura \ref{fig:cluster-tree-topology}. Cada \emph{cluster} se identifica por un \emph{cluster} ID. ZigBee no soporta la topología \emph{cluster tree} pero IEEE 802.15.4 sí la soporta.
    \begin{figure}[h]
        \centering
        \includegraphics[width=7.5cm]{cluster-tree-topology}
        \caption{Topología \emph{cluster tree}}
        \label{fig:cluster-tree-topology}
    \end{figure}
    \item \textbf{Topología de malla:}  la topología de malla, también conocida como una red punto-a-punto, consiste de un coordinador, varios \emph{routers} y dispositivos finales, como se muestra en la figura \ref{fig:mesh-topology}.
    \begin{figure}[h]
        \centering
        \includegraphics[width=7.5cm]{mesh-topology}
        \caption{Topología de malla}
        \label{fig:mesh-topology}
    \end{figure}
    Las siguientes son las características de una topología de malla:
    \begin{itemize}
        \item Una topología de malla es una red multisalto (\emph{multihop}), los paquetes pasan a través de varios saltos para alcanzar su destino.
        \item El rango de una red puede ser incrementado al añadir más dispositivos a la red.
        \item Se pueden eliminar las zonas muertas.
        \item Una topología de malla previene errores, lo que significa que durante una transmisión si una ruta falla, el nodo encontrará una ruta alterna al destino. 
        \item Los dispositivos pueden estar cerca el uno del otro de esta forma usan menos energía.
        \item Agregar o remover dispositivos es fácil.
        \item Un dispositivo se puede comunicar con cualquier otro en la red.
        \item Comparado con la topología de estrella, la topología de malla requiere mayor sobrecarga (\emph{overhead}).
        \item El enrutamiento en malla usa un protocolo de enrutamiento más complejo que una topología de estrella.
    \end{itemize}    
\end{itemize}

\subsection{Direccionamiento de dispositivos finales (nodos)}
Cuando un dispositivo se une a una red ZigBee, el coordinador ZigBee o el \emph{router} asigna un dirección lógica de 16 bits al dispositivo. También, cada dispositivo tiene una dirección IEEE de 64 bits y no hay dos dispositivos que tengan la misma dirección IEEE en el mundo entero. Las direcciones cortas (16 bits) pueden ser utilizadas por los dispositivos en la red. La ventaja de usar direcciones de 16 bits es la de extender la vida de la bateria. Una dirección de 16 bits reduce el tamaño del marco (\emph{frame}). Un tamaño de marco menor da como resultado un tiempo de transmisión más corto. Un tiempo de transmisión menor significa un mayor tiempo de bateria. La desventaja de usar una dirección de 16 bits es que dos nodos en redes diferentes pueden tener la misma dirección.

\subsection{Profundidad de una red, número de hijos y asignación de direcciones en la red}

\paragraph*{\textbf{Profundidad de una red}} La profundidad de una red está determinada por el número de \emph{routers} (saltos -- \emph{hops}) desde el coordinador hacia el dispositivo más lejano, en donde la lejanía se define por el número de saltos. En una topología de estrella, la profundidad de la red es uno.

\paragraph*{\textbf{Número de hijos}} El número de dispositivos finales (hijos) que estan conectados al \emph{router} o coordinador. El coordinador establece el número máximo de hijos conectados al \emph{router}. 

\paragraph*{\textbf{Asignación de direcciones}} En una topología estrella, cada coordinador mantiene información acerca de la red, tal y como el número máximo de hijos, el número máximo de \emph{routers} y usa esta información para asignar una dirección a cada \emph{router}. Los \emph{routers}, entonces assignan las direcciones a sus dispositivos respectivos. Sin embargo, en una topologia de malla, cada \emph{router} asigna una dirección aleatoria a sus respectivos dispositivos finales.

\subsection{Arquitectura del protocolo ZigBee}
La figura \ref{fig:zigbee-architecture} muestra la arquitectura del protocolo ZigBee. La \emph{ZigBee Alliance} desarrolló el \emph{ZigBee device object} (ZDO), la subcapa de soporte de aplicación (APS por sus siglas en inglés), la capa de red y la gestión de la seguridad. IEEE 802.15.4 es utilizada para la capa MAC y la capa física. 

\begin{figure}[h]
    \centering
    \includegraphics[width=8cm]{zigbee-architecture}
    \caption{Arquitectura del protocolo ZigBee}
    \label{fig:zigbee-architecture}
\end{figure}

La arquitectura del protocolo ZigBee está dividio en tres secciones:
\begin{enumerate}
    \item IEEE 802.15.4, que consiste de las capas MAC y física.
    \item Capas ZigBee, que consiste de la capa de red, el ZDO, la subcapa de aplicación y la gestión de la seguridad.
    \item Aplicación del fabricante: fabricantes de dispositivos ZigBee puede usasr el perfil de aplicación ZigBee o desarrollar su propio perfil de aplicación.
\end{enumerate}

\subsubsection{La capa física}
La capa física realiza modulación de las señales salientes y desmodulación de las señales entrantes. Transmite y recibe información desde una fuente. La tabla \ref{table:1} muestra la banda de frecuencia de la capa física, tasa de banda (\emph{band rate}) y el número de canal.

\begin{table}[h!]
    \centering
    \begin{tabular}{llcl}
        \toprule[1.5pt]
        Banda frecuencia & País & Tasa de banda & Números de canal\\
        \midrule
        868.3 MHz & Países europeos & 20Kbps & 0 \\
        902-928 MHz & USA & 40 Kbps & 1 - 10 \\
        2.405 MHz & Global & 250 Kbps & 11 - 26 \\
        \bottomrule[1.5pt]
    \end{tabular}
    \caption{Banda de frecuencia de la capa física}
    \label{table:1}
\end{table}

\subsubsection{Capa Media Access Control (MAC)}
Las funciones de la capa MAC son acceder a la red usando \emph{carrier-sense multiple access with collision avoidance} (CSMA/CA), para transmisión de marcos \emph{beacon (beacon frames)} para sincronización, y proveer una transmisión confiable.

\subsubsection{Capa de red}
La capa de red está localizada entre la capa MAC y la subcapa de soporte de aplicación. Provee las siguientes funciones:
\begin{itemize}
    \item Iniciar una red
    \item Gestión de los dispositivos finales que se unen o dejan la red
    \item Descubrimiento de rutas
    \item Descubrimiento de vecinos
\end{itemize}

\subsubsection{Subcapa de soport de aplicación (APS)}
La subcapa de soporte de aplicación (APS) proporciona los servicios necesarios para los objetos de aplicación (\emph{endpoints}) y el \emph{ZigBee device object} (ZDO) para interactuar con la capa de red para datos y gestión de servicios. Algunos de los servicios proporcionados por la APS a los objectos de aplicación para transferencia de datos son \emph{request, confirm \textnormal{y} response.} Además, la APS provee comunicación para las aplicaciones al definir una estructura de comunicación unificada (por ejemplo, un perfil, \emph{cluster}, o \emph{endpoint}).

\subsubsection{Objecto de aplicación (endpoint)}
Un objeto de aplicación define la entrada y salida del APS. Por ejemplo, un \emph{switch} que controla una luz es la entrada para el objeto de aplicación, y la salida es la condición de la bombilla. Cada nodo puede tener 240 objetos de aplicación separados. Un objeto de aplicación también puede ser conocido también como un \emph{endpoint}(EP). La figura \ref{fig:home-control-lighting} muestra un ejemplo de control de iluminación doméstico.

\begin{figure}[h]
    \centering
    \includegraphics[width=8cm]{home-controlling-lighting}
    \caption{Control de iluminación doméstico}
    \label{fig:home-control-ligthing}
\end{figure}

\paragraph{\emph{ZigBee device object} (ZDO)} un dispositivo ZigBee realiza control y administración de objetos de aplicación. El ZDO realiza tareas de gestión global de dispositivos:
\begin{itemize}
    \item Determina el tipo de dispositivo en una red (por ejemplo, un dispositivo final, \emph{router} o coordinador).
    \item Inicializa el APS, la capa de red, y el proveedor de servicios de seguridad.
    \item Realiza descubrimiento de dispositivos y servicios.
    \item Inicializa el coordinador para el establecimiento de una red.
    \item Gestión de seguridad.
    \item Gestión de la red.
\end{itemize}

\paragraph{Nodo final} Cada nodo o dispositivo final puede tener múltiples EPs. Cada EP contiene un perfil de aplicación, tal y como automatización del hogar, y puede ser usado para controlar varios dispositivos o uno solo. Cada EP define funciones de comunicación dentro de un dispositivo. Como se muestra en la figura \ref{fig:home-control-ligthing}, el \emph{switch} de la habitación controla la iluminación de la habitación y el control remoto es utilizado para controlar tres luces: habitación, pasillo1 y pasillo2.

\paragraph{Modo de direccionamiento ZigBee}
ZigBee usa direccionamiento directo, agrupado y de emisión (\emph{broadcast}) para la transmisión de información. In direccionamiento directo, dos dispositivos se comunican directamente entre sí. Esto requiere que el dispositivo fuente tenga ambos, la dirección y el \emph{endpoint} del dispositivo destino. El direccionamiento agrupado requiere que la applicación asigne un grupo de miembros a uno o más dispositivos. Un paquete es transmitido al grupo de direcciones en donde el dispositivo destino reside. La dirección \emph{broadcast} es usada para enviar un paquete a todos los dispositivos de la red. 


%\section{Servicios ZeegBee}
%
%\textbf{Lo vamos a desarrollar???}
%
%\subsection{ZigBee Device Object (ZDO)}
%\subsection{ZigBee Device Profile (ZDP)}
%\subsection{Zigbee Cluster Library (ZCL)}

\section{Otras tecnologías wireless}
ZigBee esta bien posicionado en el area de sensores wireless y cntrol de redes pero existen otras soluciones wirless que al igual que ZigBee utilizan el protocolo 82.15.4:

\begin{itemize}
  \item \textbf{Wireless USB:} es una tecnología emergente de dispositivos operados por batería. Es una tecnología poco costosa, genial para las baterías pero no funciona bien en gran escala y noo provee seguridad. Esta basada en Ultra Wideband (UWB) ultizada mayormente en baterías de dispositivos periféricos para PC.
  \item \textbf{WiFi:} se utiliza en sensores y control de redes, es una tecnología que comenzó con el mercado de PCs pero comienza a ganar terreno en el área de los dispositivos. Es mas cara que ZigBee donde WiFi requiere de un equipo mayor para correr todo el protocolo. 
  \item \textbf{Bluetooth:} Es un protocolo seguro, utilizado en celulares, wearables como relojes, audífonos y demás, no es una tecnología cara  pero no tiene la misma capacidad de batería que un dispositivo ZigBee.Bluetooth tiene otra limitante: no escala bien en redes amplias, siendo solo aplicable en redes de no mas de siete dispositivos.
  \item \textbf{Wibree:} Utilizado en relojes y sensores de movimiento para el cuerpo. Esta diseñada para ser de bajo consumo de poder pero, esta limitada por el numero de nodos que pueden haber en una red. Es una tecnología que sigue en desarrollo. 
  \item \textbf{Z-Wave}: Tecnología utilizada para la automatización de casas, no posee un estándar pues la compañía Zensys es la única que manufactura componentes Z-Wave. Es el competidor directo de ZigBee. 
\end{itemize}


% needed in second column of first page if using \IEEEpubid
%\IEEEpubidadjcol


% An example of a floating figure using the graphicx package.
% Note that \label must occur AFTER (or within) \caption.
% For figures, \caption should occur after the \includegraphics.
% Note that IEEEtran v1.7 and later has special internal code that
% is designed to preserve the operation of \label within \caption
% even when the captionsoff option is in effect. However, because
% of issues like this, it may be the safest practice to put all your
% \label just after \caption rather than within \caption{}.
%
% Reminder: the "draftcls" or "draftclsnofoot", not "draft", class
% option should be used if it is desired that the figures are to be
% displayed while in draft mode.
%
%\begin{figure}[!t]
%\centering
%\includegraphics[width=2.5in]{myfigure}
% where an .eps filename suffix will be assumed under latex, 
% and a .pdf suffix will be assumed for pdflatex; or what has been declared
% via \DeclareGraphicsExtensions.
%\caption{Simulation results for the network.}
%\label{fig_sim}
%\end{figure}

% Note that IEEE typically puts floats only at the top, even when this
% results in a large percentage of a column being occupied by floats.
% However, the Computer Society has been known to put floats at the bottom.


% An example of a double column floating figure using two subfigures.
% (The subfig.sty package must be loaded for this to work.)
% The subfigure \label commands are set within each subfloat command,
% and the \label for the overall figure must come after \caption.
% \hfil is used as a separator to get equal spacing.
% Watch out that the combined width of all the subfigures on a 
% line do not exceed the text width or a line break will occur.
%
%\begin{figure*}[!t]
%\centering
%\subfloat[Case I]{\includegraphics[width=2.5in]{box}%
%\label{fig_first_case}}
%\hfil
%\subfloat[Case II]{\includegraphics[width=2.5in]{box}%
%\label{fig_second_case}}
%\caption{Simulation results for the network.}
%\label{fig_sim}
%\end{figure*}
%
% Note that often IEEE papers with subfigures do not employ subfigure
% captions (using the optional argument to \subfloat[]), but instead will
% reference/describe all of them (a), (b), etc., within the main caption.
% Be aware that for subfig.sty to generate the (a), (b), etc., subfigure
% labels, the optional argument to \subfloat must be present. If a
% subcaption is not desired, just leave its contents blank,
% e.g., \subfloat[].


% An example of a floating table. Note that, for IEEE style tables, the
% \caption command should come BEFORE the table and, given that table
% captions serve much like titles, are usually capitalized except for words
% such as a, an, and, as, at, but, by, for, in, nor, of, on, or, the, to
% and up, which are usually not capitalized unless they are the first or
% last word of the caption. Table text will default to \footnotesize as
% IEEE normally uses this smaller font for tables.
% The \label must come after \caption as always.
%
%\begin{table}[!t]
%% increase table row spacing, adjust to taste
%\renewcommand{\arraystretch}{1.3}
% if using array.sty, it might be a good idea to tweak the value of
% \extrarowheight as needed to properly center the text within the cells
%\caption{An Example of a Table}
%\label{table_example}
%\centering
%% Some packages, such as MDW tools, offer better commands for making tables
%% than the plain LaTeX2e tabular which is used here.
%\begin{tabular}{|c||c|}
%\hline
%One & Two\\
%\hline
%Three & Four\\
%\hline
%\end{tabular}
%\end{table}


% Note that the IEEE does not put floats in the very first column
% - or typically anywhere on the first page for that matter. Also,
% in-text middle ("here") positioning is typically not used, but it
% is allowed and encouraged for Computer Society conferences (but
% not Computer Society journals). Most IEEE journals/conferences use
% top floats exclusively. 
% Note that, LaTeX2e, unlike IEEE journals/conferences, places
% footnotes above bottom floats. This can be corrected via the
% \fnbelowfloat command of the stfloats package.




\section{Conclusion}

[La Conclusion]





% if have a single appendix:
%\appendix[Proof of the Zonklar Equations]
% or
%\appendix  % for no appendix heading
% do not use \section anymore after \appendix, only \section*
% is possibly needed

% use appendices with more than one appendix
% then use \section to start each appendix
% you must declare a \section before using any
% \subsection or using \label (\appendices by itself
% starts a section numbered zero.)
%


%\appendices
%\section{Proof of the First Zonklar Equation}
%Appendix one text goes here.
%
%\section{}
%Appendix two text goes here.


% use section* for acknowledgment
%\ifCLASSOPTIONcompsoc
%  % The Computer Society usually uses the plural form
%  \section*{Acknowledgments}
%\else
%  % regular IEEE prefers the singular form
%  \section*{Acknowledgment}
%\fi


%The authors would like to thank...


% Can use something like this to put references on a page
% by themselves when using endfloat and the captionsoff option.
\ifCLASSOPTIONcaptionsoff
  \newpage
\fi



% trigger a \newpage just before the given reference
% number - used to balance the columns on the last page
% adjust value as needed - may need to be readjusted if
% the document is modified later
%\IEEEtriggeratref{8}
% The "triggered" command can be changed if desired:
%\IEEEtriggercmd{\enlargethispage{-5in}}

% references section

% can use a bibliography generated by BibTeX as a .bbl file
% BibTeX documentation can be easily obtained at:
% http://www.ctan.org/tex-archive/biblio/bibtex/contrib/doc/
% The IEEEtran BibTeX style support page is at:
% http://www.michaelshell.org/tex/ieeetran/bibtex/
%\bibliographystyle{IEEEtran}
% argument is your BibTeX string definitions and bibliography database(s)
%\bibliography{IEEEabrv,../bib/paper}
%
% <OR> manually copy in the resultant .bbl file
% set second argument of \begin to the number of references
% (used to reserve space for the reference number labels box)


\begin{thebibliography}{1}

\bibitem{bombal}
D.~Gislason. \emph{Zigbee Wireless Networking}. \hskip 1em plus 0.5em minus 0.4em\relax Elsevier. ISBN: 978-0-7506-85979. 2008.


\bibitem{gshewender}
A.~Gschwender, A.~Elahi. \emph{Zigbee Wireless Sensor and Control Network}. \hskip 1em plus 0.5em minus 0.4em\relax Prentice Hall. ISBN: 9780137134854. 2009.



  
\end{thebibliography}

% biography section
% 
% If you have an EPS/PDF photo (graphicx package needed) extra braces are
% needed around the contents of the optional argument to biography to prevent
% the LaTeX parser from getting confused when it sees the complicated
% \includegraphics command within an optional argument. (You could create
% your own custom macro containing the \includegraphics command to make things
% simpler here.)
%\begin{IEEEbiography}[{\includegraphics[width=1in,height=1.25in,clip,keepaspectratio]{mshell}}]{Michael Shell}
% or if you just want to reserve a space for a photo:

\begin{IEEEbiography}[{\includegraphics[width=1in,height=1.25in,clip,keepaspectratio]{priscilla-piedra}}]{Priscilla Piedra}
es Ingeniera de Computación del Tecnologíco de Costa Rica. Actualmente es estudiante del programa de Maestría en Ciencas de la Computación en la misma universidad. Sus principales intereses son: \emph{cloud computing} y automatización. 
\end{IEEEbiography}

% if you will not have a photo at all:
\begin{IEEEbiography}[{\includegraphics[width=1in,height=1.25in,clip,keepaspectratio]{martin-flores}}]{Martín Flores}
es Ingeniero en Informática de la Universidad Nacional. Actualmente, realiza sus estudios de Maestría en Ciencias de la Computación del Tecnológico de Costa Rica. Sus principales intereses son: lenguajes de programación, ingeniería de software y \emph{DevOps}.
\end{IEEEbiography}

% insert where needed to balance the two columns on the last page with
% biographies
%\newpage

%\begin{IEEEbiographynophoto}{Jane Doe}
%Biography text here.
%\end{IEEEbiographynophoto}

% You can push biographies down or up by placing
% a \vfill before or after them. The appropriate
% use of \vfill depends on what kind of text is
% on the last page and whether or not the columns
% are being equalized.

\vfill

% Can be used to pull up biographies so that the bottom of the last one
% is flush with the other column.
%\enlargethispage{-5in}



% that's all folks
\end{document}


